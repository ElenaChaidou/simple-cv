\documentclass[%
    11pt,
  oneside
  ]{memoir}

\usepackage{enumitem}
%%% Font declarations
\usepackage{fontspec} % Choose fonts intelligently
\setmainfont{EBGaramond12-Regular.otf}[%
    ItalicFont = EBGaramond12-Italic.otf ,
     % The default is not to have a bold font defined…
  BoldFont = EBGaramond12-Regular.otf ,
      BoldFeatures = {Letters=SmallCaps} ,
      Path =./fonts/ ,
  Ligatures=Historic ,
  Ligatures=Rare ,
  Mapping=tex-text ,
  ItalicFeatures={Style=Swash} ,
]
\setmonofont{SourceCodePro-Regular.otf}[
  Path =./fonts/
]

%%% PDF generation
\usepackage[ breaklinks=true, hidelinks ]{hyperref}
\hypersetup{%
            pdftitle={Henry Walton Jones, Jr.~CV},
            pdfauthor={Moacir P. de Sá Pereira, Henry Walton Jones, Jr.},
            pdfborder={0 0 0},
            breaklinks=true}

%%% Layout of the page
  \setlrmarginsandblock{.5in}{*}{*}
  \setulmarginsandblock{.5in}{*}{*}
  \setheadfoot{0pt}{\baselineskip} % footer is a baseline tall.
  \setheaderspaces{*}{0pt}{*}

%%% Define the (default) chapter style and choose the chapter
  \makechapterstyle{line}{%
    \setlength{\beforechapskip}{0pt}
    \setlength{\afterchapskip}{0pt}
    \renewcommand*{\chaptitlefont}{\Large\scshape}
    \renewcommand*{\printchaptertitle}[1]{%
    \chaptitlefont ##1 \smallskip\hrule\vspace{\baselineskip}
    }
  }
\chapterstyle{line}

%%% Section styling. We assume two, margin and overlapping.
\setsecnumdepth{subsubsection} % Keep the section counters running (see below).
% This gets complicated, because we need the numbers to roll over so that the
% subsections have an appropriate beforeskip.
  % using regular headings mode
      \setsecheadstyle{\large\scshape}
      \setaftersecskip{\baselineskip} % add a blank line after the heading
    \setlength{\parindent}{3em} % indent paragraphs
% Blank out all printed counters and their teeny spaces. Vital both for
% sections and subsections.
\makeatletter % enter into macro mode
\renewcommand{\@seccntformat}[1]{}
\makeatother % exit macro mode.

%%% Subsection styling. Should someone (me) include subsections in their CV,
% things get complex.

\setsubsecheadstyle{\normalsize} % Subheads shouldn’t be big.
\setaftersubsecskip{0em} % Subheads should have no space after them
% New problem: \subsection should have no top margin, so it’s in line w/ the 
% section header. But if that’s the case, when a subsection appears later in
% the section, it looks just like a regular line. We need subsection to have
% conditional beforeskips, based on whether it’s the first subsection or not.
\let\oldsubsection\subsection % to avoid recursion
\renewcommand{\subsection}[1]{%
  \ifnumequal{\value{subsection}}{0} % If this is the first subsection
  {\setbeforesubsecskip{0em} \oldsubsection{#1}} % make beforeskip 0
  {\setbeforesubsecskip{\medskipamount} \oldsubsection{#1}} % else add a skip
}
  \setsubsecindent{3em} % In overlapped, yes.


%%% List styling. They behave differently based on margin or regular headings.
\setlist[itemize]{nosep} % No gaps between list items.
  \setlist[itemize,1]{leftmargin=!,labelwidth=1em,labelindent=3em} % Push list in.
% And now the markers.
\renewcommand{\labelitemi}{•}
\renewcommand{\labelitemii}{-}
\renewcommand{\labelitemiii}{*}

\checkandfixthelayout

%%% Format footer
\copypagestyle{chapter}{plain}
  \makeoddhead{chapter}{}{}{}
  \makeoddfoot{chapter}{}{}{%
    Moacir P. de Sá Pereira, Henry Walton Jones, Jr.,  \today, \thepage}
\pagestyle{chapter}

%%% Redefine \section to tighten vertical space
\let\oldsection\section
\renewcommand{\section}[1]{%
  \oldsection{#1}
  \leavevmode
  \par
  \vspace{\dimexpr-\baselineskip-\parskip}
}


\begin{document}

      \chapter*{Henry Walton ``Indiana'' Jones, Jr.}
  

  \hypertarget{contact-information}{%
  \section{Contact Information}\label{contact-information}}
    \begin{minipage}[t]{0.3\textwidth}
      Brody Hall 134\\ Marshall College\\ Bedford, CT 0649x
    \end{minipage}
    \begin{minipage}[t]{0.7\textwidth}
                {\textit{Tel:}} +1 860 555-5555 \\
                        {\textit{E-mail:}} hwjones@marshall-ct.edu \\
                        {\textit{Web:}} http://archaeology.marshall-ct.edu/faculty/jones-henry/ \\
                        {\textit{Twitter:}} \href{http://twitter.com/IndianaJones}{@IndianaJones} \\
                        {\textit{GitHub:}} \href{http://github.com/IndianaJones}{@IndianaJones}
            \end{minipage}
      % \vspace{\baselineskip}
    \medskip%amount
    \par Citizenship:
    United States of America and Aramathea
  \hypertarget{fields-of-specialization}{%
\section{Fields of Specialization}\label{fields-of-specialization}}

Ancient relics, imperial theft of cultural patrimony, Grail lore,
anti-fascism, masculinist archæology, Celtic mythology, Herpetology.

\hypertarget{education}{%
\section{Education}\label{education}}

\begin{itemize}
\tightlist
\item
  \textbf{Sorbonne}, PhD, 1925
\item
  \textbf{University of Chicago}, AB, 1922, Archæology

  \begin{itemize}
  \tightlist
  \item
    Adviser: Abner Ravenwood
  \end{itemize}
\end{itemize}

\hypertarget{academic-appointments}{%
\section{Academic Appointments}\label{academic-appointments}}

\begin{itemize}
\tightlist
\item
  Marcus Brody Distinguished Service Professor, Archæology,
  \textbf{Marshall College}, Bedford, CT 1950--present
\item
  Professor, Archæology, \textbf{Barnett College}, Fairfield, NY
  1937--1950
\item
  Associate Professor, Archæology, \textbf{Marshall College}, Bedford,
  CT 1935--1937
\item
  Assistant Professor, Archæology, \textbf{Princeton University},
  1932--1935
\item
  Visiting Assistant Professor, Medieval Literature, \textbf{Princeton
  University}, 1930
\item
  Visiting Scholar, Archæology, \textbf{Marshall College}, 1925
\item
  Instructor, Celtic Mythology, \textbf{London University}, 1925--1927
\end{itemize}

\hypertarget{selected-publications}{%
\section{Selected Publications}\label{selected-publications}}

\begin{itemize}
\tightlist
\item
  Jones, Henry W. Jr. \emph{How I Got Rich off Colonialism}. Princeton,
  NJ: Princeton University Press, 1944.
\item
  Jones, Henry W. Jr. ``It Belongs in a Museum.'' \emph{Journal of
  Patrimony Exploitation} 10, no. 1 (1944), 793--843.
\end{itemize}

\hypertarget{invited-lectures}{%
\section{Invited Lectures}\label{invited-lectures}}

\begin{itemize}
\tightlist
\item
  ``How I Found the Ark.''

  \begin{itemize}
  \tightlist
  \item
    \textbf{Princeton University}, 1948
  \end{itemize}
\item
  ``How I Lost the Ark.''

  \begin{itemize}
  \tightlist
  \item
    \textbf{Princeton University}, 1942
  \end{itemize}
\end{itemize}

\hypertarget{selected-teaching}{%
\section{Selected Teaching}\label{selected-teaching}}

\begin{itemize}
\tightlist
\item
  \textbf{Marshall College}

  \begin{itemize}
  \tightlist
  \item
    Archæology 101 - Discovering the Past
  \item
    Archæology 223
  \item
    Archæology 225 - Ancient Egypt
  \end{itemize}
\item
  \textbf{Barnett College}

  \begin{itemize}
  \tightlist
  \item
    Archæology 101
  \end{itemize}
\end{itemize}

\hypertarget{service-related-professional-activities}{%
\section{Service \& Related Professional
Activities}\label{service-related-professional-activities}}

\begin{itemize}
\tightlist
\item
  Gateway Project
\item
  Boy Scouts of America
\item
  Anti-fascists in Archæology
\item
  University of Chicago Alumni Network
\end{itemize}

\hypertarget{skills}{%
\section{Skills}\label{skills}}

\begin{itemize}
\tightlist
\item
  \LaTeX
\item
  Bullwhip
\end{itemize}

\hypertarget{languages}{%
\section{Languages}\label{languages}}

\begin{itemize}
\tightlist
\item
  Native fluency: English
\item
  Near-native fluency: French, German, Italian, Spanish, Russian,
  Swedish, Greek, Arabic, Turkish, Vietnamese, Swahili, Latin, Nepalese
  and Chinese.
\item
  Reading and writing: Hindi and Sinhalese.
\item
  Foundations: Mayan and Quechua.
\end{itemize}

\end{document}

